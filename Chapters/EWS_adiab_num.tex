	\section{adiabatic number}

A natural way to say that a parameter is being sweept slowly when  the system is close to a stable attractor would be to say that, if the control parameter at $t_i$ is $\lambda_i$, then after a relaxation time $t^*_i=-1/M(\la_i)$ the system moves to a state given by  $\lambda_i+\delta \la$ on which the relaxation time has not change significantly. 

Thus we want $\Delta M=M_f-M_i$ to be small, with $M_f=M(\lambda_i+\delta \la)$, and $M_i= M(\lambda_i) $.

To quantify this we choose wither of the follow adimensional quantities:

\begin{equation}
	\begin{aligned}
		\xi_i&=\frac{\Delta M}{M_i}=\frac{M_f}{M_i}-1 \\
		\xi_f&=\frac{\Delta M}{M_f}=1-\frac{M_i}{M_f}
	\end{aligned}
	\label{eq:xi}
\end{equation}

We will call this quantities the adiabatic number of the augmented system. 

And they have the following properties:
\begin{itemize}
	\item They are positive if the system becomes locally more stable, since the relaxation times decreases. 
	\item They are zero if the system does not change stability. Thus, the are zero if the system is not being swiped.
	\item They are negative if the system losses stability.
\end{itemize}

In particular they take values in given intervals $\xi_f  \in (-\inf,1)$, $\xi_i  \in (-1,\inf)$.





\tikzmath{ int \k ; }
\begin{figure}
	\centering
%	\begin{tikzpicture}
%		\node (time) at (11,0)  [right]{$t$};
%		\draw [->,very thick,black](0,0) -- (11,0);
%		\node[above,left] at (0,3) {a)};
%		\foreach \i in {0,...,4}
%		{ \draw (2*\i,0) node[rectangle, thin,left color=col2, right color=white!,shading angle=180,anchor=south west, minimum height=3cm,minimum width=2cm,opacity=0.2+\i*(1-0.2)/9] () {}; }
%		\foreach \i in {0,...,5}
%		{ \filldraw (2*\i,0) circle [circle,black,fill=black,align=center,radius=0.05cm] ; }
%		\node[below] (a) at (0,0) [center,below]{$t_i$};
%		\node[below] (b) at (2,0) [center,below]{$t_{i+1}$};
%		\node[below] (b) at (4,0) [center,below]{$t_{i+2}$};
%		\draw [dotted , thick , black,right](6,-0.3) -- (8,-0.3);
%		\node[below] (b) at (10,0) [center,below]{$t_{n}$};
%		\tzfn[->,col1,ultra thick]{3.5*exp(-1.3*\x)}[0.1:1.5]{}[al]		
%	\end{tikzpicture}
	
	\begin{tikzpicture}
		
		\k=0;
	
		\draw (0,0) node[rectangle,  thin,left color=col2, right color=white!,shading angle=180,anchor=south west ,anchor=south west, minimum height=3cm,minimum width=1.1 cm,opacity=1] () {};
		\draw (1.1,0) node[rectangle,  thin,left color=col2, right color=white!,shading angle=180,anchor=south west ,anchor=south west, minimum height=3cm,minimum width=1.2 cm,opacity=0.9] () {};
		\node[below] (a) at (0,0) [center,below]{$t_i$};
		\node[below] (b) at (1.1,0) [center,below]{$t_{i}+t_i^*$};
%		\node[below] (b) at (2,0) [center,below]{$t_{i+2}$};
		\draw [dotted , thick , black,right](3.3,-0.3) -- (5,-0.3);

		\tzfn[->,col1,ultra thick]{3.5*exp(ln(.1/3.5)*\x)}[0.:1.1]{}[ar]
		\tzfn[->,col1,ultra thick]{3.5*exp(ln(.1/3.5)/1.1*(\x-1.1))}[1.1:2.3]{}[ar]		
%		
		\draw (6,0) node[rectangle,  thin,left color=col2, right color=white!,shading angle=180,anchor=south west ,anchor=south west, minimum height=3cm,minimum width=2cm,opacity=0.5] () {};
		\draw (8,0) node[rectangle,  thin,left color=col2, right color=white!,shading angle=180,anchor=south west ,anchor=south west, minimum height=3cm,minimum width=4cm,opacity=0.3] () {};
		\node[below] (a) at (6,0) [center,below]{$t_i$};
 		\node[below] (b) at (8,0) [center,below]{$t_{i}+t_i^*$};

		\tzfn[->,col1,ultra thick]{3.5*exp(ln(.1/3.5)/2*(\x-6))}[6:8]{}[ar]
		\tzfn[->,col1,ultra thick]{3.5*exp(ln(.1/3.5)/4*(\x-8))}[8:12]{}[ar]
		\node[above,left] at (0,3) {b)};	
		
		\node (time) at (13,0)  [right]{$t$};
		\draw [->,very thick,black](0,0) -- (13,0);
		\foreach \i in {0,1.1,2.3}
		{\filldraw (\i,0) circle [circle,black,fill=black,align=center,radius=0.05cm] ; 
		};
		\foreach \i in {6,8,12}
		{			\filldraw (\i,0) circle [circle,black,fill=black,align=center,radius=0.05cm] ; 
		}
	\end{tikzpicture}
	\caption{Scheme for the definition of slow sweeping. We consider a slow sweeping if the relaxation time does not change significantly after one relaxation time $t^*$}.
	\label{fig: velocity_param3}
\end{figure}


